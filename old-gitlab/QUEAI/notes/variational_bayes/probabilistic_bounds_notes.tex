\documentclass[10pt, a4paper]{article}
\usepackage[utf8]{inputenc}
\usepackage[T1]{fontenc}
\usepackage[english]{babel}

\usepackage{geometry}

\usepackage{amsmath, amsthm, amsfonts, amssymb}
\usepackage{cleveref}

%==============================================================================
% ABBREVIATIONS
%------------------------------------------------------------------------------
\usepackage{amsmath, amsfonts, amssymb}
\RequirePackage{bbm}

% utility
\newcommand{\til}{\widetilde}
\newcommand{\N}{\mathbb{N}}
\newcommand{\Z}{\mathbb{Z}}
\newcommand{\R}{\mathbb{R}}
\newcommand{\C}{\mathbb{C}}
\newcommand{\E}{\mathbb{E}}
\newcommand{\V}{\mathbb{V}}
\renewcommand{\P}{\mathbb{P}}
\newcommand{\1}{\mathbbm{1}}
\DeclareMathOperator*{\supp}{supp}
\DeclareMathOperator*{\argmin}{argmin}
\DeclareMathOperator*{\argmax}{argmax}
\DeclareMathOperator*{\minimize}{minimize}
\DeclareMathOperator*{\maximize}{maximize}
\DeclareMathOperator*{\KL}{KL}
\newcommand{\kron}{\otimes}
\DeclareMathOperator*{\diag}{diag}
\newcommand{\D}{\ensuremath{\mathrm{d}}}    % for integrals, as in \d x for dx
\newcommand\topS{\rule{0pt}{2.6ex}}         % Top strut for tabular
\newcommand\botS{\rule[-1.2ex]{0pt}{0pt}}   % Bottom strut for tabular

% mathbf
\newcommand{\bfa}{\mathbf{a}}
\newcommand{\bfb}{\mathbf{b}}
\newcommand{\bfc}{\mathbf{c}}
\newcommand{\bfd}{\mathbf{d}}
\newcommand{\bfe}{\mathbf{e}}
\newcommand{\bff}{\mathbf{f}}
\newcommand{\bfg}{\mathbf{g}}
\newcommand{\bfh}{\mathbf{h}}
\newcommand{\bfi}{\mathbf{i}}
\newcommand{\bfj}{\mathbf{j}}
\newcommand{\bfk}{\mathbf{k}}
\newcommand{\bfl}{\mathbf{l}}
\newcommand{\bfm}{\mathbf{m}}
\newcommand{\bfn}{\mathbf{n}}
\newcommand{\bfo}{\mathbf{o}}
\newcommand{\bfp}{\mathbf{p}}
\newcommand{\bfq}{\mathbf{q}}
\newcommand{\bfr}{\mathbf{r}}
\newcommand{\bfs}{\mathbf{s}}
\newcommand{\bft}{\mathbf{t}}
\newcommand{\bfu}{\mathbf{u}}
\newcommand{\bfv}{\mathbf{v}}
\newcommand{\bfw}{\mathbf{w}}
\newcommand{\bfx}{\mathbf{x}}
\newcommand{\bfy}{\mathbf{y}}
\newcommand{\bfz}{\mathbf{z}}

\newcommand{\bfA}{\mathbf{A}}
\newcommand{\bfB}{\mathbf{B}}
\newcommand{\bfC}{\mathbf{C}}
\newcommand{\bfD}{\mathbf{D}}
\newcommand{\bfE}{\mathbf{E}}
\newcommand{\bfF}{\mathbf{F}}
\newcommand{\bfG}{\mathbf{G}}
\newcommand{\bfH}{\mathbf{H}}
\newcommand{\bfI}{\mathbf{I}}
\newcommand{\bfJ}{\mathbf{J}}
\newcommand{\bfK}{\mathbf{K}}
\newcommand{\bfL}{\mathbf{L}}
\newcommand{\bfM}{\mathbf{M}}
\newcommand{\bfN}{\mathbf{N}}
\newcommand{\bfO}{\mathbf{O}}
\newcommand{\bfP}{\mathbf{P}}
\newcommand{\bfQ}{\mathbf{Q}}
\newcommand{\bfR}{\mathbf{R}}
\newcommand{\bfS}{\mathbf{S}}
\newcommand{\bfT}{\mathbf{T}}
\newcommand{\bfU}{\mathbf{U}}
\newcommand{\bfV}{\mathbf{V}}
\newcommand{\bfW}{\mathbf{W}}
\newcommand{\bfX}{\mathbf{X}}
\newcommand{\bfY}{\mathbf{Y}}
\newcommand{\bfZ}{\mathbf{Z}}


% mathcal
\newcommand{\CA}{\mathcal{A}}
\newcommand{\CB}{\mathcal{B}}
\newcommand{\CC}{\mathcal{C}}
\newcommand{\CD}{\mathcal{D}}
\newcommand{\CE}{\mathcal{E}}
\newcommand{\CF}{\mathcal{F}}
\newcommand{\CG}{\mathcal{G}}
\newcommand{\CH}{\mathcal{H}}
\newcommand{\CI}{\mathcal{I}}
\newcommand{\CJ}{\mathcal{J}}
\newcommand{\CK}{\mathcal{K}}
\newcommand{\CL}{\mathcal{L}}
\newcommand{\CM}{\mathcal{M}}
\newcommand{\CN}{\mathcal{N}}
\newcommand{\CO}{\mathcal{O}}
\newcommand{\CP}{\mathcal{P}}
\newcommand{\CQ}{\mathcal{Q}}
\newcommand{\CR}{\mathcal{R}}
\newcommand{\CS}{\mathcal{S}}
\newcommand{\CT}{\mathcal{T}}
\newcommand{\CU}{\mathcal{U}}
\newcommand{\CV}{\mathcal{V}}
\newcommand{\CW}{\mathcal{W}}
\newcommand{\CX}{\mathcal{X}}
\newcommand{\CY}{\mathcal{Y}}
\newcommand{\CZ}{\mathcal{Z}}


\title{Probabilistic Bounds for the Interval Method}
\author{placeholder}

%%%%% %%%%% %%%%% %%%%% %%%%%
\begin{document}

\maketitle

\section*{Loss}

For some data distribution $X,Y$ and some regularization parameter $\beta$ we use the following loss:
\begin{align*}
\mathcal{E} = \int_X \E\left[\text{max}(Y_x-\overline{y}_x,0)^2\right]+\E\left[\text{max}(\underline{y}_x-Y_x,0)^2\right]+\beta (\overline{y}_x-\underline{y}_x)\ d\P_X.
\end{align*}
Assuming during training this loss is optimized yields
\begin{align*}
0 &= \int_X \frac{\partial }{\partial \overline{y}_x} \left( \E\left[\text{max}(Y_x-\overline{y}_x,0)^2\right]+\E\left[\text{max}(\underline{y}_x-Y_x,0)^2\right]+\beta (\overline{y}_x-\underline{y}_x)\right) \ d\P_X \\
&= -\int_X 2\E\left[\text{max}(Y_x-\overline{y}_x,0)\right]\ d\P_X+\beta\\
\Rightarrow \frac{1}{2}\beta &= \int_X \E\left[\text{max}(Y_x-\overline{y}_x,0)\right]\ d\P_X.
\intertext{Analogously}
\frac{1}{2}\beta &= \int_X \E\left[\text{max}(\underline{y}_x-Y_x,0)\right]\ d\P_X.
\end{align*}

\section*{Using the Markow Inequality}
Using the Markow Inequality with $h_1(\zeta):= \text{max}(\zeta-\overline{y}_x,0)$ and $h_2(\zeta):= \text{max}(\zeta+\underline{y}_x,0)$ we can see that for the marginalized distribution $Y_x$ the following holds:
\begin{align*}
&\P(Y_x \geq \overline{y}_x+\lambda \beta)  \leq \frac{\E\left[h_1(Y_x) \right]}{h_1(\overline{y}_x+\lambda \beta)} = \frac{\E\left[\text{max}(Y_x-\overline{y}_x,0) \right]}{\lambda \beta}
\intertext{and}
&\P(Y_x \leq \underline{y}_x-\lambda \beta)\\ = &\P(-Y_x \geq -\underline{y}_x+\lambda \beta) \leq \frac{\E\left[h_2(-Y_x) \right]}{h_2(-\underline{y}_x+\lambda \beta)} = \frac{\E\left[\text{max}(\underline{y}_x-Y_x,0) \right]}{\lambda \beta}.
\end{align*}
Using the previous equation we can now see the following:
\begin{align*}
&\P(\left\lbrace\text{Label is inside interval bounds plus $\lambda \beta$} \right\rbrace)\\=&\int_X \P(\underline{y}_x-\lambda \beta\leq Y_x \leq \overline{y}_x+\lambda \beta)\ d\P_X
= 1-\int_X \P(Y_x \leq \underline{y}_x-\lambda \beta)+ \P(Y_x \geq \overline{y}_x+\lambda \beta)\ d\P_X
\end{align*}
with
\begin{align*}
\int_X \P(Y_x \leq \underline{y}_x-\lambda \beta)+ \P(Y_x \geq \overline{y}_x+\lambda \beta)\ d\P_X &\leq \int_X \frac{\E\left[\text{max}(Y_x-\overline{y}_x,0) \right]}{\lambda \beta}+\frac{\E\left[\text{max}(\underline{y}_x-Y_x,0) \right]}{\lambda \beta} d\P_X\\
&= \frac{1}{\lambda}.
\end{align*}
Therefore
\begin{align*}
&\P(\left\lbrace\text{Label is inside interval bounds plus $\lambda \beta$} \right\rbrace) \geq 1-\frac{1}{\lambda}.
\end{align*}
We can furthermore bound the probability that for a given data point $x$ the label $y_x$ has a probability of more than $\alpha$ to be outside the interval bounds:
\begin{align*}
\frac{1}{\lambda} &\geq \E_X\left[\P( Y_x < \underline{y}_x-\lambda \beta,\ Y_x > \overline{y}_x+\lambda \beta) \right] \geq \E_X\left[\alpha \1_{\P( Y_x < \underline{y}_x-\lambda \beta,\ Y_x > \overline{y}_x+\lambda \beta)> \alpha} \right]\\
\Rightarrow \frac{1}{\lambda \alpha} &\geq \E_X\left[\1_{\P( Y_x < \underline{y}_x-\lambda \beta,\ Y_x > \overline{y}_x+\lambda \beta)> \alpha} \right] = \E_X\left[1-\1_{\P( Y_x < \underline{y}_x-\lambda \beta,\ Y_x > \overline{y}_x+\lambda \beta)leq \alpha} \right]\\
&= 1-\E_X\left[\1_{\P(\underline{y}_x-\lambda \beta\leq Y_x \leq \overline{y}_x+\lambda \beta)\geq 1-\alpha} \right]\\
\Rightarrow 1-\frac{1}{\lambda \alpha} &\leq \E_X\left[\1_{\P(\underline{y}_x-\lambda \beta\leq Y_x \leq \overline{y}_x+\lambda \beta)\geq 1-\alpha} \right]
\end{align*}
In other words, for $\lambda>0$ and $\alpha>0$ the probability mass of all samples $x$ where corresponding label $y_x$ has the probability of at least $1-\alpha$ to be inside the interval is at least $ 1-\frac{1}{\lambda \alpha}$.


\end{document}
